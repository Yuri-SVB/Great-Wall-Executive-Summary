% values.tex -- central registry of model inputs (CLEAN ROOM).
% Edit only these numbers to update the model. Keep names stable to avoid breaking formulas.

%% ================= Computation Economics =================
% Electricity cost per kWh (USD).
\newcommand{\electricityCostPerKwh}{0.12}

% Average power consumption during computation (watts).
\newcommand{\computePowerWatts}{350}

% Number of weekly job runs per month.
\newcommand{\weeklyRunsPerMonth}{4.28}

% Patrick's markup multipliers by tier (progressive).
\newcommand{\patrickMarkupBasic}{2.5}
\newcommand{\patrickMarkupMedium}{2.6}
\newcommand{\patrickMarkupProfessional}{3.5}  % Premium for 48hr commitment
\newcommand{\patrickMarkupGolden}{2.8}
\newcommand{\patrickMarkupPlatinum}{2.9}

% Mark's platform fee percentage (of Charlie's payment).
\newcommand{\markPlatformFeePercent}{28}

% Computation job durations (hours).
\newcommand{\computeHoursBasic}{2}
\newcommand{\computeHoursMedium}{24}
\newcommand{\computeHoursProfessional}{48}
\newcommand{\computeHoursGolden}{168}
\newcommand{\computeHoursPlatinum}{336}

%% ================= Derived Computation Costs =================
% Patrick's monthly electricity cost per tier (USD).
\newcommand{\patrickCostBasic}{\fpeval{\computeHoursBasic * \computePowerWatts / 1000 * \electricityCostPerKwh * \weeklyRunsPerMonth}}
\newcommand{\patrickCostMedium}{\fpeval{\computeHoursMedium * \computePowerWatts / 1000 * \electricityCostPerKwh * \weeklyRunsPerMonth}}
\newcommand{\patrickCostProfessional}{\fpeval{\computeHoursProfessional * \computePowerWatts / 1000 * \electricityCostPerKwh * \weeklyRunsPerMonth}}
\newcommand{\patrickCostGolden}{\fpeval{\computeHoursGolden * \computePowerWatts / 1000 * \electricityCostPerKwh * \weeklyRunsPerMonth}}
\newcommand{\patrickCostPlatinum}{\fpeval{\computeHoursPlatinum * \computePowerWatts / 1000 * \electricityCostPerKwh * \weeklyRunsPerMonth}}

% Patrick's gross revenue per tier (after markup, USD/month).
\newcommand{\patrickGrossRevenueBasic}{\fpeval{\patrickCostBasic * {1 + \patrickMarkupBasic}}}
\newcommand{\patrickGrossRevenueMedium}{\fpeval{\patrickCostMedium * {1 + \patrickMarkupMedium}}}
\newcommand{\patrickGrossRevenueProfessional}{\fpeval{\patrickCostProfessional * {1 + \patrickMarkupProfessional}}}
\newcommand{\patrickGrossRevenueGolden}{\fpeval{\patrickCostGolden * {1 + \patrickMarkupGolden}}}
\newcommand{\patrickGrossRevenuePlatinum}{\fpeval{\patrickCostPlatinum * {1 + \patrickMarkupPlatinum}}}

%% ================= Subscription Tiers =================
% COMPUTATION MARKETPLACE TIERS:
% Tiers represent different computation job durations (hours):
% Basic = 2 hours, Medium = 24 hours, Professional = 48 hours,
% Golden = 168 hours, Platinum = 336 hours
% All jobs recur weekly (4.28 times per month)

% Subscription tier mix (% of subscribers).
\newcommand{\subBasicMix}{35}

% Subscription tier monthly price (USD).
% 2-hour computation jobs @ 4.28/month
% Patrick cost: $0.36/mo, Markup: 7.2x
\newcommand{\subBasicPrice}{2.60}

% Subscription tier mix (% of subscribers).
\newcommand{\subMediumMix}{40}

% Subscription tier monthly price (USD).
% 24-hour computation jobs @ 4.28/month
% Patrick cost: $4.32/mo, Markup: 6.9x
\newcommand{\subMediumPrice}{30.00}

% Subscription tier mix (% of subscribers) - NEW TIER
% Professional tier bridges the gap between Medium and Golden
\newcommand{\subProfessionalMix}{15}

% Subscription tier monthly price (USD) - NEW TIER
% 48-hour computation jobs @ 4.28/month
% Patrick cost: $8.64/mo, Markup: 10.4x (premium for longer commitment)
\newcommand{\subProfessionalPrice}{90.00}

% Subscription tier mix (% of subscribers).
\newcommand{\subGoldenMix}{9}

% Subscription tier monthly price (USD).
% 168-hour (1 week) computation jobs @ 4.28/month
% Patrick cost: $30.24/mo, Markup: 7.3x
\newcommand{\subGoldenPrice}{220.00}

% Subscription tier mix (% of subscribers) - NEW TIER
% Platinum tier for enterprise/whale customers
% Expected <1% adoption but important for completeness
\newcommand{\subPlatinumMix}{1}

% Subscription tier monthly price (USD).
% 336-hour (2 week) computation jobs @ 4.28/month
% Patrick cost: $60.40/mo, Markup: 7.3x
\newcommand{\subPlatinumPrice}{440.00}

% Patrick's gross profit per tier (after markup, USD/month).
\newcommand{\patrickGrossProfitBasic}{\fpeval{\patrickCostBasic * \patrickMarkupBasic}}
\newcommand{\patrickGrossProfitMedium}{\fpeval{\patrickCostMedium * \patrickMarkupMedium}}
\newcommand{\patrickGrossProfitProfessional}{\fpeval{\patrickCostProfessional * \patrickMarkupProfessional}}
\newcommand{\patrickGrossProfitGolden}{\fpeval{\patrickCostGolden * \patrickMarkupGolden}}
\newcommand{\patrickGrossProfitPlatinum}{\fpeval{\patrickCostPlatinum * \patrickMarkupPlatinum}}

% Patrick's profit margin per tier (USD/month).
\newcommand{\patrickProfitBasic}{\fpeval{\patrickGrossRevenueBasic - \patrickCostBasic}}
\newcommand{\patrickProfitMedium}{\fpeval{\patrickGrossRevenueMedium - \patrickCostMedium}}
\newcommand{\patrickProfitProfessional}{\fpeval{\patrickGrossRevenueProfessional - \patrickCostProfessional}}
\newcommand{\patrickProfitGolden}{\fpeval{\patrickGrossRevenueGolden - \patrickCostGolden}}
\newcommand{\patrickProfitPlatinum}{\fpeval{\patrickGrossRevenuePlatinum - \patrickCostPlatinum}}

% Mark's revenue per tier (platform fee, USD/month).
\newcommand{\markRevenueBasic}{\fpeval{\subBasicPrice * \markPlatformFeePercent / 100}}
\newcommand{\markRevenueMedium}{\fpeval{\subMediumPrice * \markPlatformFeePercent / 100}}
\newcommand{\markRevenueProfessional}{\fpeval{\subProfessionalPrice * \markPlatformFeePercent / 100}}
\newcommand{\markRevenueGolden}{\fpeval{\subGoldenPrice * \markPlatformFeePercent / 100}}
\newcommand{\markRevenuePlatinum}{\fpeval{\subPlatinumPrice * \markPlatformFeePercent / 100}}

% Total markup per tier (Charlie price / Patrick cost).
\newcommand{\totalMarkupBasic}{\fpeval{\subBasicPrice / \patrickCostBasic}}
\newcommand{\totalMarkupMedium}{\fpeval{\subMediumPrice / \patrickCostMedium}}
\newcommand{\totalMarkupProfessional}{\fpeval{\subProfessionalPrice / \patrickCostProfessional}}
\newcommand{\totalMarkupGolden}{\fpeval{\subGoldenPrice / \patrickCostGolden}}
\newcommand{\totalMarkupPlatinum}{\fpeval{\subPlatinumPrice / \patrickCostPlatinum}}

%% ================= Hardware SKUs =================
% Hardware gross margin (%).
\newcommand{\hwEntryMargin}{40}

% Hardware customer mix (% of hardware buyers).
\newcommand{\hwEntryMix}{60}

% Hardware SKU price (USD).
\newcommand{\hwEntryPrice}{500}

% Hardware gross margin (%).
\newcommand{\hwPremiumMargin}{50}

% Hardware customer mix (% of hardware buyers).
\newcommand{\hwPremiumMix}{10}

% Hardware SKU price (USD).
\newcommand{\hwPremiumPrice}{2000}

% Hardware gross margin (%).
\newcommand{\hwProfMargin}{45}

% Hardware customer mix (% of hardware buyers).
\newcommand{\hwProfMix}{30}

% Hardware SKU price (USD).
\newcommand{\hwProfPrice}{1000}

%% ================= Merchandise =================
% Merchandise attach rate for subscription customers (%).
% Industry benchmark: 15-25% for typical SaaS (Shopify 2024 report)
% We project 60% - admittedly optimistic but justified by:
% 1. Security-focused community has strong identity/tribal aspects
% 2. Bitcoin users historically show high brand loyalty (see Ledger, Trezor merch adoption)
% 3. Physical security tools create natural cross-sell opportunity for branded items
% 4. "Not your keys, not your coins" culture drives physical product affinity
% 5. Anonymous marketplace users value privacy-respecting branded gear
\newcommand{\merchAttachRate}{60}

% Merchandise attach rate for hardware customers (%).
% Higher than subscription due to:
% 1. Hardware buyers already demonstrated willingness to purchase physical goods
% 2. Natural bundling opportunity at point of hardware purchase
% 3. Security-conscious users value branded accessories for their devices
\newcommand{\hwMerchAttachRate}{65}

% Merchandise gross margin (%).
% Based on print-on-demand industry standards (Printful 2023):
% - T-shirts: 50% margin typical
% - Accessories: 45-55% margin range
% - Weighted average across product mix: 48%
\newcommand{\merchAvgMargin}{48}

% Merchandise average basket price (USD).
% Calculated from typical security/crypto merchandise pricing:
% - T-shirts: $25 (vs $20-30 industry)
% - Hoodies: $45 (vs $40-60 industry)
% - Mugs/accessories: $15-20
% - Average basket contains 1.2 items @ $23 each = $28 total
\newcommand{\merchAvgPrice}{28}

%% ================= Merchandise Product Details =================
% T-shirt price and margin
\newcommand{\merchTshirtPrice}{25}
\newcommand{\merchTshirtMargin}{50}

% Caps/Hats price and margin
\newcommand{\merchCapPrice}{20}
\newcommand{\merchCapMargin}{45}

% Mugs price and margin
\newcommand{\merchMugPrice}{15}
\newcommand{\merchMugMargin}{55}

% Stickers/Decals price and margin
\newcommand{\merchStickerPrice}{5}
\newcommand{\merchStickerMargin}{70}

% Hoodies/Coats price and margin
\newcommand{\merchHoodiePrice}{45}
\newcommand{\merchHoodieMargin}{40}

% Backpacks price and margin
\newcommand{\merchBackpackPrice}{35}
\newcommand{\merchBackpackMargin}{45}

%% ================= Customer Metrics =================
% Hardware customers in given year (count).
\newcommand{\hwCustomersYearThree}{2000}

% Hardware customers in given year (count).
\newcommand{\hwCustomersYearTwo}{1000}

% New subscribers in given year (count).
\newcommand{\newSubsYearOne}{10000}

% New subscribers in given year (count).
\newcommand{\newSubsYearThree}{10000}

% New subscribers in given year (count).
\newcommand{\newSubsYearTwo}{10000}

% Total active subscribers at end of year (count).
% CORRECTED for 5% annual churn (95% retention)
\newcommand{\totalSubsYearOne}{10000}

% Total active subscribers at end of year (count).
% Year 2: 10,000 × 0.95 + 10,000 new = 19,500
\newcommand{\totalSubsYearTwo}{19500}

% Total active subscribers at end of year (count).
% Year 3: 19,500 × 0.955 + 10,000 new = 28,623
\newcommand{\totalSubsYearThree}{28623}

%% ================= Churn =================
% ANNUAL churn rate (%).
\newcommand{\churnYearOne}{5}

% ANNUAL churn rate (%).
\newcommand{\churnYearThree}{4}

% ANNUAL churn rate (%).
\newcommand{\churnYearTwo}{4.5}

%% ================= CAC & Offsets =================
% Customer Acquisition Cost component (USD per acquired customer).
\newcommand{\cacContent}{18}

% Customer Acquisition Cost component (USD per acquired customer).
\newcommand{\cacDigital}{20}

% Customer Acquisition Cost component (USD per acquired customer).
\newcommand{\cacEvents}{60}

%% ================= Market Size Assumptions =================
% Serviceable Available Market (count).
\newcommand{\samHw}{200000}

% Serviceable Available Market (count).
\newcommand{\samSubs}{500000}

% Total Addressable Market (count).
\newcommand{\tamHwGlobal}{2000000}

% Total Addressable Market (count).
\newcommand{\tamSubsGlobal}{5000000}

% Target penetration of SAM (%).
\newcommand{\targetShareHw}{5}

% Target penetration of SAM (%).
\newcommand{\targetShareSubs}{10}

%% ================= Valuation Multiples =================
% Valuation multiple for ARR (x).
\newcommand{\arrMultiple}{3}

% Valuation multiple for hardware gross profit (x).
\newcommand{\hwMultiple}{1.5}

% Conservative revenue multiple for Year 2 valuation.
\newcommand{\yearTwoRevMultiple}{2.5}

%% ================= Funding Plan =================
% Seed round target raise (USD).
\newcommand{\seedAmount}{200000}

% Seed investor equity offered (%).
\newcommand{\seedEquity}{3.5}

% Seed pre-money valuation (USD).
\newcommand{\seedValuation}{5700000}

% Series A target raise (USD).
\newcommand{\seriesAAmount}{600000}

% Series A investor equity offered (%).
\newcommand{\seriesAEquity}{4}

% Series A pre-money valuation (USD).
\newcommand{\seriesAValuation}{17000000}

%% ================= Target Valuations =================
% Year 3 target valuation low (USD millions).
\newcommand{\targetValLow}{30}

% Year 3 target valuation high (USD millions).
\newcommand{\targetValHigh}{35}

% Year 3 optimistic valuation low (USD millions).
\newcommand{\optimisticValLow}{50}

% Year 3 optimistic valuation high (USD millions).
\newcommand{\optimisticValHigh}{60}

%% ================= Budget =================
% Annual budget allocation (USD).
\newcommand{\budgetContent}{25000}

% Annual budget allocation (USD).
\newcommand{\budgetDigital}{60000}

% Annual budget allocation (USD).
\newcommand{\budgetEvents}{15000}

%% ================= Use of Proceeds - Seed Round =================
% Subscription platform development (USD).
\newcommand{\seedUseSubPlatform}{50000}

% Marketing and sales (USD).
\newcommand{\seedUseMarketing}{100000}

% Working capital for merchandise (USD).
\newcommand{\seedUseWorkingCap}{20000}

% Operations and team (USD).
\newcommand{\seedUseOperations}{30000}

%% ================= Use of Proceeds - Series A =================
% Hardware development (USD).
\newcommand{\seriesAUseHwDev}{250000}

% Manufacturing setup (USD).
\newcommand{\seriesAUseMfg}{100000}

% Marketing expansion (USD).
\newcommand{\seriesAUseMarketing}{100000}

% Team growth (USD).
\newcommand{\seriesAUseTeam}{150000}

%% ================= Operational Metrics =================
% Beta launch target users.
\newcommand{\betaUsers}{1000}

% Initial team size for seed round.
\newcommand{\seedTeamSize}{2}

% Runway in months for seed round.
\newcommand{\seedRunway}{12}

% CAC reduction percentage from merchandise.
\newcommand{\cacReductionPercent}{40}

% Combined TAM in thousands.
\newcommand{\combinedTAMk}{700}

% Target customer count for subscriptions (thousands).
\newcommand{\targetSubsCountk}{50}

% Target customer count for hardware (thousands).
\newcommand{\targetHwCountk}{10}

%% ================= Gross Margins =================
% Subscription gross margin (%).
% 85% reflects typical SaaS margins after Patrick's payout
% Mark retains ~25-30% of Charlie's payment as platform fee
\newcommand{\subGrossMargin}{85}

%% ================= Timeline Ranges =================
% Beta launch timing (months)
\newcommand{\betaLaunchStartMonth}{4}
\newcommand{\betaLaunchEndMonth}{6}

% Hardware development timing (months)
\newcommand{\hwDesignStartMonth}{13}
\newcommand{\hwDesignEndMonth}{17}

% Security certification timing (months)
\newcommand{\hwCertStartMonth}{18}
\newcommand{\hwCertEndMonth}{19}

% Marketplace infrastructure build (months)
\newcommand{\marketplaceBuildStartMonth}{1}
\newcommand{\marketplaceBuildEndMonth}{3}

% Marketplace scaling period (months)
\newcommand{\marketplaceScaleStartMonth}{7}
\newcommand{\marketplaceScaleEndMonth}{12}

%% ================= DERIVED CALCULATIONS =================
%% These are calculated from the base values above.
%% Do not edit these directly - they update automatically.

% Subscription weighted average monthly revenue per user (USD).
\newcommand{\subWeightedAvgMonthly}{\fpeval{\subBasicPrice * \subBasicMix / 100 + \subMediumPrice * \subMediumMix / 100 + \subProfessionalPrice * \subProfessionalMix / 100 + \subGoldenPrice * \subGoldenMix / 100 + \subPlatinumPrice * \subPlatinumMix / 100}}

% Subscription weighted average annual revenue per user (USD).
\newcommand{\subWeightedAvgAnnual}{\fpeval{\subWeightedAvgMonthly * 12}}

% Hardware weighted average gross profit per unit (USD).
\newcommand{\hwWeightedAvgGP}{\fpeval{(\hwEntryPrice * \hwEntryMargin / 100) * \hwEntryMix / 100 + (\hwProfPrice * \hwProfMargin / 100) * \hwProfMix / 100 + (\hwPremiumPrice * \hwPremiumMargin / 100) * \hwPremiumMix / 100}}

% Hardware weighted average price per unit (USD).
\newcommand{\hwWeightedAvgPrice}{\fpeval{\hwEntryPrice * \hwEntryMix / 100 + \hwProfPrice * \hwProfMix / 100 + \hwPremiumPrice * \hwPremiumMix / 100}}

% Hardware weighted average margin (%).
\newcommand{\hwWeightedAvgMargin}{\fpeval{\hwWeightedAvgGP / \hwWeightedAvgPrice * 100}}

% Merchandise average profit per unit (USD).
\newcommand{\merchAvgProfit}{\fpeval{\merchAvgPrice * \merchAvgMargin / 100}}

% Merchandise offset for subscription customers (USD).
\newcommand{\merchOffsetSubsCalc}{\fpeval{\merchAttachRate / 100 * \merchAvgProfit}}

% Merchandise offset for hardware customers (USD).
\newcommand{\merchOffsetHwCalc}{\fpeval{\hwMerchAttachRate / 100 * \merchAvgProfit}}

% Subscription annual revenue calculations based on corrected subscriber counts
\newcommand{\subRevenueYearOne}{\fpeval{\subWeightedAvgAnnual * \totalSubsYearOne}}
\newcommand{\subRevenueYearTwo}{\fpeval{\subWeightedAvgAnnual * \totalSubsYearTwo}}
\newcommand{\subRevenueYearThree}{\fpeval{\subWeightedAvgAnnual * \totalSubsYearThree}}

% ARR calculations - based on exit MRR × 12
% ARR = Active subscribers at year-end × average monthly revenue × 12
\newcommand{\subARRYearOne}{\fpeval{\totalSubsYearOne * \subWeightedAvgMonthly * 12}}
\newcommand{\subARRYearTwo}{\fpeval{\totalSubsYearTwo * \subWeightedAvgMonthly * 12}}
\newcommand{\subARRYearThree}{\fpeval{\totalSubsYearThree * \subWeightedAvgMonthly * 12}}

% LTV Calculation Components
% Average churn rate across 3 years
\newcommand{\avgAnnualChurn}{\fpeval{(\churnYearOne + \churnYearTwo + \churnYearThree) / 3 / 100}}

% Theoretical customer lifetime in years (1/churn_rate)
\newcommand{\theoreticalLifetimeYears}{\fpeval{1 / \avgAnnualChurn}}

% Conservative LTV cap in years (industry standard)
\newcommand{\ltvCapYears}{7}

% Actual LTV calculation years (minimum of theoretical and cap)
\newcommand{\ltvYearsUsed}{\fpeval{min(\theoreticalLifetimeYears, \ltvCapYears)}}

% Annual gross profit per subscriber
\newcommand{\subAnnualGrossProfit}{\fpeval{\subWeightedAvgAnnual * \subGrossMargin / 100}}

% Final LTV calculation
\newcommand{\subLTV}{\fpeval{\subAnnualGrossProfit * \ltvYearsUsed}}

% Monthly gross profit (for payback calculation)
\newcommand{\subMonthlyGrossProfit}{\fpeval{\subWeightedAvgMonthly * \subGrossMargin / 100}}
