% values.tex -- central registry of model inputs (CLEAN ROOM).
% Edit only these numbers to update the model. Keep names stable to avoid breaking formulas.

%% ================= Computation Economics =================
% Electricity cost per kWh (USD).
\newcommand{\electricityCostPerKwh}{0.12}

% Average power consumption during computation (watts).
\newcommand{\computePowerWatts}{350}

% Number of weekly job runs per month.
\newcommand{\weeklyRunsPerMonth}{4.28}

% Patrick's markup multipliers by tier (progressive).
\newcommand{\patrickMarkupBasic}{2.5}
\newcommand{\patrickMarkupMedium}{3.0}
\newcommand{\patrickMarkupProfessional}{3.5}  % Premium for 48hr commitment
\newcommand{\patrickMarkupGolden}{5.0}
\newcommand{\patrickMarkupPlatinum}{10.0}

% Mark's platform fee percentage (of Charlie's payment).
\newcommand{\markPlatformFeePercent}{28}

% Computation job durations (hours).
\newcommand{\computeHoursBasic}{2}
\newcommand{\computeHoursMedium}{24}
\newcommand{\computeHoursProfessional}{48}
\newcommand{\computeHoursGolden}{168}
\newcommand{\computeHoursPlatinum}{336}

%% ================= Patrick's Cost Calculations =================
% Patrick's monthly electricity cost per tier (USD).
\newcommand{\patrickCostBasic}{\fpeval{\computeHoursBasic * \computePowerWatts / 1000 * \electricityCostPerKwh * \weeklyRunsPerMonth}}
\newcommand{\patrickCostMedium}{\fpeval{\computeHoursMedium * \computePowerWatts / 1000 * \electricityCostPerKwh * \weeklyRunsPerMonth}}
\newcommand{\patrickCostProfessional}{\fpeval{\computeHoursProfessional * \computePowerWatts / 1000 * \electricityCostPerKwh * \weeklyRunsPerMonth}}
\newcommand{\patrickCostGolden}{\fpeval{\computeHoursGolden * \computePowerWatts / 1000 * \electricityCostPerKwh * \weeklyRunsPerMonth}}
\newcommand{\patrickCostPlatinum}{\fpeval{\computeHoursPlatinum * \computePowerWatts / 1000 * \electricityCostPerKwh * \weeklyRunsPerMonth}}

% Patrick's gross revenue per tier (what Patrick receives from Mark, USD/month).
\newcommand{\patrickGrossBasic}{\fpeval{\patrickCostBasic * \patrickMarkupBasic}}
\newcommand{\patrickGrossMedium}{\fpeval{\patrickCostMedium * \patrickMarkupMedium}}
\newcommand{\patrickGrossProfessional}{\fpeval{\patrickCostProfessional * \patrickMarkupProfessional}}
\newcommand{\patrickGrossGolden}{\fpeval{\patrickCostGolden * \patrickMarkupGolden}}
\newcommand{\patrickGrossPlatinum}{\fpeval{\patrickCostPlatinum * \patrickMarkupPlatinum}}

%% ================= Subscription Tiers =================
% COMPUTATION MARKETPLACE TIERS:
% Tiers represent different computation job durations (hours):
% Basic = 2 hours, Medium = 24 hours, Professional = 48 hours,
% Golden = 168 hours, Platinum = 336 hours
% All jobs recur weekly (4.28 times per month)

% Subscription tier mix (% of subscribers).
\newcommand{\subBasicMix}{35}

% Subscription tier monthly price (USD).
% 2-hour computation jobs @ 4.28/month
% Calculated: Patrick receives (cost × markup), Mark adds platform fee
\newcommand{\subBasicPrice}{\fpeval{round(\patrickGrossBasic / (1 - \markPlatformFeePercent/100), 0)}}

% Subscription tier mix (% of subscribers).
\newcommand{\subMediumMix}{40}

% Subscription tier monthly price (USD).
% 24-hour computation jobs @ 4.28/month
% Calculated: Patrick receives (cost × markup), Mark adds platform fee
\newcommand{\subMediumPrice}{\fpeval{round(\patrickGrossMedium / (1 - \markPlatformFeePercent/100), 0)}}

% Subscription tier mix (% of subscribers) - NEW TIER
% Professional tier bridges the gap between Medium and Golden
\newcommand{\subProfessionalMix}{15}

% Subscription tier monthly price (USD) - NEW TIER
% 48-hour computation jobs @ 4.28/month
% Calculated: Patrick receives (cost × markup), Mark adds platform fee
\newcommand{\subProfessionalPrice}{\fpeval{round(\patrickGrossProfessional / (1 - \markPlatformFeePercent/100), 0)}}

% Subscription tier mix (% of subscribers).
\newcommand{\subGoldenMix}{9}

% Subscription tier monthly price (USD).
% 168-hour (1 week) computation jobs @ 4.28/month
% Calculated: Patrick receives (cost × markup), Mark adds platform fee
\newcommand{\subGoldenPrice}{\fpeval{round(\patrickGrossGolden / (1 - \markPlatformFeePercent/100), 0)}}

% Subscription tier mix (% of subscribers) - NEW TIER
% Platinum tier for enterprise/whale customers
% Expected <1% adoption but important for completeness
\newcommand{\subPlatinumMix}{1}

% Subscription tier monthly price (USD) - NEW TIER
% 336-hour (2 week) computation jobs @ 4.28/month  
% Calculated: Patrick receives (cost × markup), Mark adds platform fee
\newcommand{\subPlatinumPrice}{\fpeval{round(\patrickGrossPlatinum / (1 - \markPlatformFeePercent/100), 0)}}

%% ================= Derived Marketplace Economics =================
% Patrick's profit margin per tier (USD/month).
\newcommand{\patrickProfitBasic}{\fpeval{\patrickGrossBasic - \patrickCostBasic}}
\newcommand{\patrickProfitMedium}{\fpeval{\patrickGrossMedium - \patrickCostMedium}}
\newcommand{\patrickProfitProfessional}{\fpeval{\patrickGrossProfessional - \patrickCostProfessional}}
\newcommand{\patrickProfitGolden}{\fpeval{\patrickGrossGolden - \patrickCostGolden}}
\newcommand{\patrickProfitPlatinum}{\fpeval{\patrickGrossPlatinum - \patrickCostPlatinum}}

% Mark's revenue per tier (platform fee, USD/month).
\newcommand{\markRevenueBasic}{\fpeval{\subBasicPrice * \markPlatformFeePercent / 100}}
\newcommand{\markRevenueMedium}{\fpeval{\subMediumPrice * \markPlatformFeePercent / 100}}
\newcommand{\markRevenueProfessional}{\fpeval{\subProfessionalPrice * \markPlatformFeePercent / 100}}
\newcommand{\markRevenueGolden}{\fpeval{\subGoldenPrice * \markPlatformFeePercent / 100}}
\newcommand{\markRevenuePlatinum}{\fpeval{\subPlatinumPrice * \markPlatformFeePercent / 100}}

% Total markup per tier (Charlie price / Patrick cost).
\newcommand{\totalMarkupBasic}{\fpeval{\subBasicPrice / \patrickCostBasic}}
\newcommand{\totalMarkupMedium}{\fpeval{\subMediumPrice / \patrickCostMedium}}
\newcommand{\totalMarkupProfessional}{\fpeval{\subProfessionalPrice / \patrickCostProfessional}}
\newcommand{\totalMarkupGolden}{\fpeval{\subGoldenPrice / \patrickCostGolden}}
\newcommand{\totalMarkupPlatinum}{\fpeval{\subPlatinumPrice / \patrickCostPlatinum}}

%% ================= Hardware SKUs =================
% Hardware gross margin (%).
\newcommand{\hwEntryMargin}{40}

% Hardware customer mix (% of hardware buyers).
\newcommand{\hwEntryMix}{60}

% Hardware SKU price (USD).
\newcommand{\hwEntryPrice}{500}

% Hardware gross margin (%).
\newcommand{\hwPremiumMargin}{50}

% Hardware customer mix (% of hardware buyers).
\newcommand{\hwPremiumMix}{10}

% Hardware SKU price (USD).
\newcommand{\hwPremiumPrice}{2000}

% Hardware gross margin (%).
\newcommand{\hwProfMargin}{45}

% Hardware customer mix (% of hardware buyers).
\newcommand{\hwProfMix}{30}

% Hardware SKU price (USD).
\newcommand{\hwProfPrice}{1000}

%% ================= Merchandise =================
% Merchandise attach rate for subscription customers (%).
% Industry benchmark: 15-25% for typical SaaS (Shopify 2024 report)
% We project 60% - admittedly optimistic but justified by:
% 1. Security-focused community has strong identity/tribal aspects
% 2. Bitcoin users historically show high brand loyalty (see Ledger, Trezor merch adoption)
% 3. Physical security tools create natural cross-sell opportunity for branded items
% 4. "Not your keys, not your coins" culture drives physical product affinity
% 5. Anonymous marketplace users value privacy-respecting branded gear
\newcommand{\merchAttachRate}{60}

% Merchandise attach rate for hardware customers (%).
% Higher than subscription due to:
% 1. Hardware buyers already demonstrated willingness to purchase physical goods
% 2. Natural bundling opportunity at point of hardware purchase
% 3. Security-conscious users value branded accessories for their devices
\newcommand{\hwMerchAttachRate}{65}

% Merchandise gross margin (%).
% Based on print-on-demand industry standards (Printful 2023):
% - T-shirts: 50% margin typical
% - Accessories: 45-55% margin range
% - Weighted average across product mix: 48%
\newcommand{\merchAvgMargin}{48}

% Merchandise average basket price (USD).
% Calculated from typical security/crypto merchandise pricing:
% - T-shirts: $25 (vs $20-30 industry)
% - Hoodies: $45 (vs $40-60 industry)
% - Mugs/accessories: $15-20
% - Average basket contains 1.2 items @ $23 each = $28 total
\newcommand{\merchAvgPrice}{28}

%% ================= Merchandise Product Details =================
% T-shirt price and margin
\newcommand{\merchTshirtPrice}{25}
\newcommand{\merchTshirtMargin}{50}

% Caps/Hats price and margin
\newcommand{\merchCapPrice}{20}
\newcommand{\merchCapMargin}{45}

% Mugs price and margin
\newcommand{\merchMugPrice}{15}
\newcommand{\merchMugMargin}{55}

% Stickers/Decals price and margin
\newcommand{\merchStickerPrice}{5}
\newcommand{\merchStickerMargin}{70}

% Hoodies/Coats price and margin
\newcommand{\merchHoodiePrice}{45}
\newcommand{\merchHoodieMargin}{40}

% Backpacks price and margin
\newcommand{\merchBackpackPrice}{35}
\newcommand{\merchBackpackMargin}{45}

%% ================= Customer Metrics =================
% New subscribers in given year (includes viral growth).
\newcommand{\newSubsYearOne}{\fpeval{round(\baseNewSubsYearOne * \viralMultiplierYearOne, 0)}}
\newcommand{\newSubsYearTwo}{\fpeval{round(\baseNewSubsYearTwo * \viralMultiplierYearTwo, 0)}}
\newcommand{\newSubsYearThree}{\fpeval{round(\baseNewSubsYearThree * \viralMultiplierYearThree, 0)}}

% Hardware customers in given year (includes viral growth).
\newcommand{\hwCustomersYearTwo}{\fpeval{round(\baseHwCustomersYearTwo * \viralMultiplierYearTwo, 0)}}
\newcommand{\hwCustomersYearThree}{\fpeval{round(\baseHwCustomersYearThree * \viralMultiplierYearThree, 0)}}

% Total active subscribers at end of year (count).
% Year 1: New subscribers only
\newcommand{\totalSubsYearOne}{\fpeval{round(\newSubsYearOne, 0)}}

% Year 2: Year 1 × retention + new subscribers
\newcommand{\totalSubsYearTwo}{\fpeval{round(\totalSubsYearOne * (1 - \churnYearOne/100) + \newSubsYearTwo, 0)}}

% Year 3: Year 2 × retention + new subscribers
\newcommand{\totalSubsYearThree}{\fpeval{round(\totalSubsYearTwo * (1 - \churnYearTwo/100) + \newSubsYearThree, 0)}}

%% ================= Churn =================
% ANNUAL churn rate (%).
\newcommand{\churnYearOne}{5}

% ANNUAL churn rate (%).
\newcommand{\churnYearThree}{4}

% ANNUAL churn rate (%).
\newcommand{\churnYearTwo}{4.5}

%% ================= Viral Marketing & Network Effects =================
% Provider referral coefficient (providers incentivized to bring clients)
% Higher coefficient since providers aren't anonymous and can publicly promote
% Based on Airbnb host referral data (HBS 2016): 0.2-0.3 range
\newcommand{\providerViralCoeff}{0.25}

% Annual viral growth factor (compounds as network grows)
% Conservative based on NFX (2018) marketplace research
\newcommand{\viralGrowthFactor}{1.10}

% CAC reduction from viral effects by year (%)
% Based on CB Insights (2021) marketplace platform studies
% Conservative estimates aligned with Kumar et al. (2007) WoM research
\newcommand{\cacViralReductionYearOne}{8}
\newcommand{\cacViralReductionYearTwo}{18}
\newcommand{\cacViralReductionYearThree}{30}

%% ================= Annual Marketing Budgets =================
% Year 1 marketing budget (from seed round)
\newcommand{\marketingBudgetYearOne}{80000}

% Year 2 marketing budget (seed remainder + Series A portion)
\newcommand{\marketingBudgetYearTwo}{120000}

% Year 3 marketing budget (Series A remainder + revenue reinvestment)
\newcommand{\marketingBudgetYearThree}{150000}

%% ================= Customer Acquisition Efficiency =================
% Customers acquired per $1000 marketing spend (improves with scale)
\newcommand{\custPerThousandYearOne}{100}
\newcommand{\custPerThousandYearTwo}{75}  % Decreases as early adopters exhausted
\newcommand{\custPerThousandYearThree}{60}

%% ================= Base Customer Acquisition (before viral) =================
% Base new customers from paid acquisition (marketing budget / CAC)
\newcommand{\baseNewSubsYearOne}{\fpeval{round(\marketingBudgetYearOne / \cacDigital, 0)}}
\newcommand{\baseNewSubsYearTwo}{\fpeval{round(\marketingBudgetYearTwo / \cacDigital, 0)}}
\newcommand{\baseNewSubsYearThree}{\fpeval{round(\marketingBudgetYearThree / \cacDigital, 0)}}

% Base hardware customers from paid acquisition
\newcommand{\baseHwCustomersYearTwo}{800}
\newcommand{\baseHwCustomersYearThree}{1500}

%% ================= Viral Customer Growth =================
% Viral multiplier per year (1 + viral coefficient × existing base × adoption rate)
% Year 1: Limited viral effect as network builds
\newcommand{\viralMultiplierYearOne}{\fpeval{1 + \providerViralCoeff * 0.2}}

% Year 2: Growing viral effect with proven providers
\newcommand{\viralMultiplierYearTwo}{\fpeval{1 + \providerViralCoeff * 0.5}}

% Year 3: Strong viral effect with established network
\newcommand{\viralMultiplierYearThree}{\fpeval{1 + \providerViralCoeff * 0.8}}

%% ================= CAC & Offsets =================
% Customer Acquisition Cost component (USD per acquired customer).
\newcommand{\cacContent}{18}

% Customer Acquisition Cost component (USD per acquired customer).
\newcommand{\cacDigital}{20}

% Customer Acquisition Cost component (USD per acquired customer).
\newcommand{\cacEvents}{60}

%% ================= Adjusted CAC with Viral Effects =================
% Effective CAC after viral reduction
\newcommand{\effectiveCACYearOne}{\fpeval{\cacDigital * (1 - \cacViralReductionYearOne/100)}}
\newcommand{\effectiveCACYearTwo}{\fpeval{\cacDigital * (1 - \cacViralReductionYearTwo/100)}}
\newcommand{\effectiveCACYearThree}{\fpeval{\cacDigital * (1 - \cacViralReductionYearThree/100)}}

%% ================= Market Size Assumptions =================
% Serviceable Available Market (count).
\newcommand{\samHw}{200000}

% Serviceable Available Market (count).
\newcommand{\samSubs}{500000}

% Total Addressable Market (count).
\newcommand{\tamHwGlobal}{2000000}

% Total Addressable Market (count).
\newcommand{\tamSubsGlobal}{5000000}

% Target penetration of SAM (%).
\newcommand{\targetShareHw}{5}

% Target penetration of SAM (%).
\newcommand{\targetShareSubs}{10}

%% ================= Valuation Multiples =================
% Valuation multiple for ARR (x).
\newcommand{\arrMultiple}{3}

% Valuation multiple for hardware gross profit (x).
\newcommand{\hwMultiple}{1.5}

% Conservative revenue multiple for Year 2 valuation.
\newcommand{\yearTwoRevMultiple}{2.5}

%% ================= Valuation Benchmarking =================
% Growth rate calculations for valuation adjustments
\newcommand{\growthRateYearOneTwo}{\fpeval{(\subARRYearTwo / \subARRYearOne - 1) * 100}}
\newcommand{\growthRateYearTwoThree}{\fpeval{(\subARRYearThree / \subARRYearTwo - 1) * 100}}

% Rule of 40 score (Growth Rate + Profit Margin)
\newcommand{\ruleOfFortyScore}{\fpeval{\growthRateYearTwoThree + \subGrossMargin}}

% Additional valuation calculations for hardware business
\newcommand{\hwValuationYearTwo}{\fpeval{\hwWeightedAvgGP * \hwCustomersYearTwo * \hwMultiple}}
\newcommand{\hwValuationYearThree}{\fpeval{\hwWeightedAvgGP * \hwCustomersYearThree * \hwMultiple}}

%% ================= Funding Plan =================
% Seed round target raise (USD).
\newcommand{\seedAmount}{200000}

% Seed investor equity offered (%).
\newcommand{\seedEquity}{3.5}

% Pre-revenue valuation factors (not formula-based)
% Based on: Team (2 devs), TAM ($5M), Business model (marketplace 28% take)
% Comparable: Pre-revenue marketplace seeds typically $2-8M
% Reality: This is negotiated, not calculated
\newcommand{\seedPreMoneyNegotiated}{5500000}

% Seed pre-money valuation (USD).
% PRE-REVENUE: Cannot use ARR multiple, must use market comparables
\newcommand{\seedValuation}{\seedPreMoneyNegotiated}

% Series A target raise (USD).
\newcommand{\seriesAAmount}{600000}

% Series A investor equity offered (%).
\newcommand{\seriesAEquity}{4}

% Series A ARR multiple (typical: 2.5-3x for growing marketplaces)
\newcommand{\seriesAARRMultiple}{2.8}

% Series A pre-money valuation (USD).
% NOW we can use ARR multiple since we have revenue
\newcommand{\seriesAValuation}{\fpeval{round(\subARRYearOne * \seriesAARRMultiple, -5)}}

%% ================= Target Valuations =================
% Conservative ARR multiple for Year 3 target (typical: 2.5-3x)
\newcommand{\targetARRMultiple}{2.5}

% Premium ARR multiple for Year 3 optimistic (typical: 3.5-4.5x for high growth)
\newcommand{\optimisticARRMultiple}{4.0}

% Year 3 target valuation low (USD millions).
% Conservative multiple on Year 3 ARR
\newcommand{\targetValLow}{\fpeval{round(\subARRYearThree * \targetARRMultiple / 1000000, 0)}}

% Year 3 target valuation high (USD millions).
% Standard multiple on Year 3 ARR
\newcommand{\targetValHigh}{\fpeval{round(\subARRYearThree * \arrMultiple / 1000000, 0)}}

% Year 3 optimistic valuation low (USD millions).
% Premium multiple on Year 3 ARR
\newcommand{\optimisticValLow}{\fpeval{round(\subARRYearThree * 3.5 / 1000000, 0)}}

% Year 3 optimistic valuation high (USD millions).
% High premium multiple on Year 3 ARR
\newcommand{\optimisticValHigh}{\fpeval{round(\subARRYearThree * \optimisticARRMultiple / 1000000, 0)}}

%% ================= Budget =================
% Annual budget allocation (USD).
\newcommand{\budgetContent}{25000}

% Annual budget allocation (USD).
\newcommand{\budgetDigital}{60000}

% Annual budget allocation (USD).
\newcommand{\budgetEvents}{15000}

%% ================= Use of Proceeds - Seed Round =================
% Subscription platform development (USD).
\newcommand{\seedUseSubPlatform}{50000}

% Marketing and sales (USD).
\newcommand{\seedUseMarketing}{100000}

% Working capital for merchandise (USD).
\newcommand{\seedUseWorkingCap}{20000}

% Operations and team (USD).
\newcommand{\seedUseOperations}{30000}

%% ================= Use of Proceeds - Series A =================
% Hardware development (USD).
\newcommand{\seriesAUseHwDev}{250000}

% Manufacturing setup (USD).
\newcommand{\seriesAUseMfg}{100000}

% Marketing expansion (USD).
\newcommand{\seriesAUseMarketing}{100000}

% Team growth (USD).
\newcommand{\seriesAUseTeam}{150000}

%% ================= Operational Metrics =================
% Beta launch target users.
\newcommand{\betaUsers}{1000}

% Initial team size for seed round.
\newcommand{\seedTeamSize}{2}

% Runway in months for seed round.
\newcommand{\seedRunway}{12}

% CAC reduction percentage from merchandise.
\newcommand{\cacReductionPercent}{40}

% Combined TAM in thousands.
\newcommand{\combinedTAMk}{700}

% Target customer count for subscriptions (thousands).
\newcommand{\targetSubsCountk}{50}

% Target customer count for hardware (thousands).
\newcommand{\targetHwCountk}{10}

%% ================= Gross Margins =================
% Subscription gross margin (%).
% Mark keeps 100% of the platform fee (28% of Charlie's payment) as gross margin
% since Mark has minimal direct costs for matchmaking
\newcommand{\subGrossMargin}{95}

%% ================= Mark's Revenue Calculations =================
% Mark's weighted average monthly revenue per user (USD).
\newcommand{\markWeightedAvgMonthly}{\fpeval{\markRevenueBasic * \subBasicMix / 100 + \markRevenueMedium * \subMediumMix / 100 + \markRevenueProfessional * \subProfessionalMix / 100 + \markRevenueGolden * \subGoldenMix / 100 + \markRevenuePlatinum * \subPlatinumMix / 100}}

% Mark's weighted average annual revenue per user (USD).
\newcommand{\markWeightedAvgAnnual}{\fpeval{\markWeightedAvgMonthly * 12}}

%% ================= Timeline Ranges =================
% Beta launch timing (months)
\newcommand{\betaLaunchStartMonth}{4}
\newcommand{\betaLaunchEndMonth}{6}

% Hardware development timing (months)
\newcommand{\hwDesignStartMonth}{13}
\newcommand{\hwDesignEndMonth}{17}

% Security certification timing (months)
\newcommand{\hwCertStartMonth}{18}
\newcommand{\hwCertEndMonth}{19}

% Marketplace infrastructure build (months)
\newcommand{\marketplaceBuildStartMonth}{1}
\newcommand{\marketplaceBuildEndMonth}{3}

% Marketplace scaling period (months)
\newcommand{\marketplaceScaleStartMonth}{7}
\newcommand{\marketplaceScaleEndMonth}{12}

%% ================= DERIVED CALCULATIONS =================
%% These are calculated from the base values above.
%% Do not edit these directly - they update automatically.

% Subscription weighted average monthly revenue per user (USD).
% This is Charlie's payment amount, not Mark's revenue
\newcommand{\subWeightedAvgMonthly}{\fpeval{\subBasicPrice * \subBasicMix / 100 + \subMediumPrice * \subMediumMix / 100 + \subProfessionalPrice * \subProfessionalMix / 100 + \subGoldenPrice * \subGoldenMix / 100 + \subPlatinumPrice * \subPlatinumMix / 100}}

% Subscription weighted average annual revenue per user (USD).
% This is Charlie's payment amount, not Mark's revenue
\newcommand{\subWeightedAvgAnnual}{\fpeval{\subWeightedAvgMonthly * 12}}

% Hardware weighted average gross profit per unit (USD).
\newcommand{\hwWeightedAvgGP}{\fpeval{(\hwEntryPrice * \hwEntryMargin / 100) * \hwEntryMix / 100 + (\hwProfPrice * \hwProfMargin / 100) * \hwProfMix / 100 + (\hwPremiumPrice * \hwPremiumMargin / 100) * \hwPremiumMix / 100}}

% Hardware weighted average price per unit (USD).
\newcommand{\hwWeightedAvgPrice}{\fpeval{\hwEntryPrice * \hwEntryMix / 100 + \hwProfPrice * \hwProfMix / 100 + \hwPremiumPrice * \hwPremiumMix / 100}}

% Hardware weighted average margin (%).
\newcommand{\hwWeightedAvgMargin}{\fpeval{\hwWeightedAvgGP / \hwWeightedAvgPrice * 100}}

% Merchandise average profit per unit (USD).
\newcommand{\merchAvgProfit}{\fpeval{\merchAvgPrice * \merchAvgMargin / 100}}

% Merchandise offset for subscription customers (USD).
\newcommand{\merchOffsetSubsCalc}{\fpeval{\merchAttachRate / 100 * \merchAvgProfit}}

% Merchandise offset for hardware customers (USD).
\newcommand{\merchOffsetHwCalc}{\fpeval{\hwMerchAttachRate / 100 * \merchAvgProfit}}

% Subscription annual revenue calculations based on corrected subscriber counts
% This is MARK's revenue (platform fees only), not total Charlie payments
\newcommand{\subRevenueYearOne}{\fpeval{\markWeightedAvgAnnual * \totalSubsYearOne}}
\newcommand{\subRevenueYearTwo}{\fpeval{\markWeightedAvgAnnual * \totalSubsYearTwo}}
\newcommand{\subRevenueYearThree}{\fpeval{\markWeightedAvgAnnual * \totalSubsYearThree}}

% ARR calculations - based on exit MRR × 12
% ARR = Active subscribers at year-end × Mark's average monthly revenue × 12
\newcommand{\subARRYearOne}{\fpeval{\totalSubsYearOne * \markWeightedAvgMonthly * 12}}
\newcommand{\subARRYearTwo}{\fpeval{\totalSubsYearTwo * \markWeightedAvgMonthly * 12}}
\newcommand{\subARRYearThree}{\fpeval{\totalSubsYearThree * \markWeightedAvgMonthly * 12}}

% LTV Calculation Components
% Average churn rate across 3 years
\newcommand{\avgAnnualChurn}{\fpeval{(\churnYearOne + \churnYearTwo + \churnYearThree) / 3 / 100}}

% Theoretical customer lifetime in years (1/churn_rate)
\newcommand{\theoreticalLifetimeYears}{\fpeval{1 / \avgAnnualChurn}}

% Conservative LTV cap in years (industry standard)
\newcommand{\ltvCapYears}{7}

% Actual LTV calculation years (minimum of theoretical and cap)
\newcommand{\ltvYearsUsed}{\fpeval{min(\theoreticalLifetimeYears, \ltvCapYears)}}

% Annual gross profit per subscriber (for Mark's business)
\newcommand{\subAnnualGrossProfit}{\fpeval{\markWeightedAvgAnnual * \subGrossMargin / 100}}

% Final LTV calculation (for Mark's business)
\newcommand{\subLTV}{\fpeval{\subAnnualGrossProfit * \ltvYearsUsed}}

% Monthly gross profit (for payback calculation - Mark's business)
\newcommand{\subMonthlyGrossProfit}{\fpeval{\markWeightedAvgMonthly * \subGrossMargin / 100}}
