% revenue-timing.tex - Actual revenue calculations accounting for acquisition timing
% Location: values/financial/revenue-timing.tex

%% ================= Revenue Timing Fundamentals =================
% For linear subscriber acquisition throughout the year:
% Month 1 subs pay for 12 months, Month 2 for 11 months, etc.
% Average months paid = (12+11+10+...+1)/12 = 6.5 months

% Average revenue months for Year 1 (linear acquisition)
\newcommand{\avgRevenueMonthsYearOne}{6.5}

% For Year 2: Starting subs pay 12 months + new subs average 6.5 months
% For Year 3: Same pattern

%% ================= Actual Revenue Calculations =================
% Year 1: All new subscribers, linear acquisition
\newcommand{\actualSubRevenueYearOne}{\fpeval{round(\newSubsYearOne * \markWeightedAvgAnnual * \avgRevenueMonthsYearOne / 12, 0)}}

% Year 2: Retained Y1 subs (full year) + new subs (partial year)
\newcommand{\retainedSubsYearTwo}{\fpeval{round(\totalSubsYearOne * (1 - \churnYearOne/100), 0)}}
\newcommand{\actualSubRevenueYearTwo}{\fpeval{round(\retainedSubsYearTwo * \markWeightedAvgAnnual + \newSubsYearTwo * \markWeightedAvgAnnual * \avgRevenueMonthsYearOne / 12, 0)}}

% Year 3: Retained Y2 subs (full year) + new subs (partial year)
\newcommand{\retainedSubsYearThree}{\fpeval{round(\totalSubsYearTwo * (1 - \churnYearTwo/100), 0)}}
\newcommand{\actualSubRevenueYearThree}{\fpeval{round(\retainedSubsYearThree * \markWeightedAvgAnnual + \newSubsYearThree * \markWeightedAvgAnnual * \avgRevenueMonthsYearOne / 12, 0)}}

%% ================= Hardware Revenue (unchanged - one-time sales) =================
\newcommand{\actualHwRevenueYearTwo}{\fpeval{round(\hwWeightedAvgGP * \hwCustomersYearTwo, 0)}}
\newcommand{\actualHwRevenueYearThree}{\fpeval{round(\hwWeightedAvgGP * \hwCustomersYearThree, 0)}}

%% ================= Total Actual Revenue =================
\newcommand{\totalActualRevenueYearOne}{\actualSubRevenueYearOne}
\newcommand{\totalActualRevenueYearTwo}{\fpeval{\actualSubRevenueYearTwo + \actualHwRevenueYearTwo}}
\newcommand{\totalActualRevenueYearThree}{\fpeval{\actualSubRevenueYearThree + \actualHwRevenueYearThree}}

%% ================= Actual Net Income/Loss =================
\newcommand{\actualNetIncomeYearOne}{\fpeval{round(\totalActualRevenueYearOne - \totalOpexYearOne - \marketingBudgetYearOne, 0)}}
\newcommand{\actualNetIncomeYearTwo}{\fpeval{round(\totalActualRevenueYearTwo - \totalOpexYearTwo - \marketingBudgetYearTwo, 0)}}
\newcommand{\actualNetIncomeYearThree}{\fpeval{round(\totalActualRevenueYearThree - \totalOpexYearThree - \marketingBudgetYearThree, 0)}}

%% ================= Recalculated Breakeven =================
% Monthly revenue per subscriber
\newcommand{\monthlyRevPerSub}{\markWeightedAvgMonthly}

% Calculate cumulative revenue vs cumulative costs month by month
% Assuming linear acquisition: N subscribers per month
\newcommand{\subsPerMonth}{\fpeval{\totalSubsYearOne / 12}}

% Breakeven occurs when cumulative revenue = cumulative costs
% Cumulative revenue at month M = sum(n * monthlyRev * (M-n+1)) for n=1 to M
% This equals: subsPerMonth * monthlyRevPerSub * M * (M+1) / 2
% Cumulative costs at month M = monthlyBurn * M

% Solving: subsPerMonth * monthlyRevPerSub * M * (M+1) / 2 = monthlyBurn * M
% Simplifies to: M = (2 * monthlyBurn) / (subsPerMonth * monthlyRevPerSub) - 1

\newcommand{\breakevenMonthActual}{\fpeval{round((2 * \monthlyBurnYearOne) / (\subsPerMonth * \monthlyRevPerSub) - 1, 0)}}

% If result is negative or too high, set a reasonable cap
\newcommand{\breakevenMonth}{\fpeval{min(max(\breakevenMonthActual, 12), 36)}}

% Subscribers at breakeven (using actual month)
\newcommand{\breakevenSubscribers}{\fpeval{round(\breakevenMonth * \subsPerMonth, -10)}}
